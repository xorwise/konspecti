\documentclass[a4paper]{article}
\setlength\parindent{1.5em}

\usepackage{indentfirst}
\usepackage[utf8]{inputenc}
\usepackage[T1]{fontenc}
\usepackage{textcomp}
\usepackage[russian]{babel}
\usepackage{amsmath, amssymb}
\usepackage{graphicx}


% figure support
\usepackage{import}
\usepackage{xifthen}
\pdfminorversion=7
\usepackage{pdfpages}
\usepackage{transparent}
\newcommand{\incfig}[1]{%
  \def\svgwidth{\columnwidth}
  \import{./figures/}{#1.pdf_tex}
}

\pdfsuppresswarningpagegroup=1
\title{Физика}
\author{Ситников Михаил}

\begin{document}
\maketitle
\tableofcontents
\newpage
\section{Электростатика}
Электростатика изучает заряды, в том числе если они неподвижны.
\paragraph{Электрический заряд}
\begin{itemize}
  \itemЭлементарный заряд -- +-e
  $e = 1.6 * 10^{-19}$Кл

\end{itemize}
\paragraph{Закон сохранения электрического заряда}
\ldots
\paragraph{Закон Кулона}
Сила электростатичнского взаимодействия F двух точечных неподвижных зарядов, находящихся в вакууме, прямо пропорциональна произведению этих зарядов $q_1$ и $q_2$, обратно пропорциональна квадрату расстояния r между зарядами и направлена вдоль соединяющей прямой.
\begin{center}
  $F = k* \frac{|q_1q_2|}{r^2}$
\end{center}
\\ В системе СИ $k= \frac{1}{4\pi \varepsilon_0}$
\\ $\varepsilon_0 = 8.85*10^{-12} \frac{Ф}{м}$ -- электрическая постоянная
\\ $k = 9*10^{9} \frac{м}{Ф}$
\paragraph{Закон Кулона в векторной форме}
\par $\vec{F}_{12} = k \frac{q_1q_2}{r^2} \frac{\vec{r}_{12}}{r}$

\par $\vec{F}_{12}$ -- сила, действующая на заряд $q_1$ со стороны заряда $q_2$
\paragraph{Принцип суперпозиции}
\begin{itemize}
  \item Результирующая сила F, действующая на заряд $q_a$ со стороны N других зарядов $q_1, q_2, \ldots q_N$ 
    \begin{center}
      $\vec{F} = \sum_{i=1}^{N} \vec{F}_{ai}$
    \end{center}
\end{itemize}
\paragraph{Напряженность электрического поля}
Силовая характеристика электрического поля
\\ $\vec{E} = \frac{\vec{F}}{q_{пр}}$ 
\\ Равна отношению силы F, действующей со стороны поля на неподвижный точечный пробный электрический заряд, помещенный в рассматриваемую точку поля,к этому заряду $q_{пр}$.

\paragraph{Линии напряженности поля}
\begin{itemize}
  \item Линии, проведенный в поле так, что касательные к нип в каждой точке совпадают по напрвлению с вектором напряженности поля.
  \item Линия напряженности считаетсяс направленной так же, как вектор $\vec{E}$ поля в рассматриваемой точке линии.
  \item Линии напряженности не пересекаются.
  \item Силовые линии начинаются и заканчиваются только на зарядах, или уходя и приходят в бесконечность.
\end{itemize}
\paragraph{Принцип суперпозиции для напряженности поля}
\begin{itemize}
  Напряженность электрического поля системы точечных зарядов равна сумме напряженностей полей каждого из этих зарядов в отдельности.
\end{itemize}
\begin{center}
  $\vec{E} = \sum_{i} \vec{E}_i$
\end{center}
\paragraph{Работа в электрическом поле}
\begin{itemize}
  $dA = \vec{F}d\vec{l} = q_0\vec{E}d\vec{l}$
\\ $ A_{12} = q_0 \int_{1}^{2} \vec{E}d\vec{l} = q_0 \int_{1}^{2} Edl\cos(\alpha)$\\
$A_{12} = \frac{qq_0}{4\pi \varepsilon_0} \int_{r_i}^{r_2} \frac{dr}{r^2} = \frac{qq_0}{4\pi \varepsilon_0} \left( \frac{1}{r_1} - \frac{1}{r_2} \right) $
\\ $ W_p = \frac{qq_0}{4\pi \varepsilon_0}$ 

\end{itemize}

\paragraph{Потенциал}
$\varphi = \frac{W_p}{q_0}$ 
\par Энергетическая характеристика электрического поля.
\\ Физическая величина, определяемая отношением потенциальной жнергией взаимодействия заряда с полем к величине этого заряда.
\\ $[\varphi] =$ Дж/Кл = В
\paragraph{Принцип суперпозиции для потенциала}
\begin{center}
  
$\varphi = \sum_{n=1}^{N} \varphi_i = \frac{1}{4\pi \varepsilon_0}\sum_{n=1}^{N} \frac{q_i}{r_i}$
\end{center}
 
\paragraph{Работа сил при пермещение заряда q из точки 1 в точку 2}
\begin{center}
$A_{12} = q(\varphi_1 - \varphi_2)$
\end{center}
\paragraph{Еще одно определение потенциала}
$\varphi = \frac{A_{\infty}}{q}$ 
\par Потенциал численно равен отношению работы, которуб совершают сиды поля над зарядом приудалении его из данной точки на бесконечностьЮ к величине этого заряда.

\par Эквипотенциальная поверхность -- вооброжаемая поверхность, все точки которой имеют одинаковый потенциал.
\paragraph{Электрическоп поле непрерывного распределения зарядов}
\begin{itemize}
  \item Линейная плотность заряда -- заряд, приходящийся наединицу длины.
    \begin{center}
      $\tau = \frac{dq}{dl}$
    \end{center}
  \itemПоверхностная плотность заряда -- заряд, приходящийся на единицу поверхности.
  \begin{center}
    $\sigma = \frac{dq}{dS}$
  \end{center}
  \itemОбъемная плотность заряда -- заряд, приходящийся на единицу объема.
  \begin{center}
    $\rho = \frac{dq}{dV}$
  \end{center}
\end{itemize}
\end{document}

